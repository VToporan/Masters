\section{Introduction}

Stereovision systems are based on the triangulation of 3-dimensional points from their projection
on the 2-dimensional planes of images captured by distinct cameras, taking inspiration from the 
stereoptic mechanism of human vision. For such cases, the determining factor in most computations
is the distance between the cameral, as it is directly linked to the maximum scene depth that can 
be perceived without noise perturbations.~\cite{withGeneralDepth, withMain, withNavigation}

In other cases however, the baseline case is that of insect stereopsis, with multiple instances of the
same scene being recorded as fractions of the same underling image. In such cases, the systems makes
use of a micro lens array in order to focus light into specific points of the image sensor, leading
to a small distance between the focal points of each image, thus losing accuracy for points that are
deeper into the scene.~\cite{withInsects}

This paper presents a variation on the latter, opting however to use different sensing areas in order
to record the image data. The setup is based on the hemispherically curved image sensor CurvIS, in
order to add a 3-dimensional component to the sensor pattern. In the end, the sensing areas can be 
combined into a larger sensor, depending on the resolution and accuracy needs of each application.